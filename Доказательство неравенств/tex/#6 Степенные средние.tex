\documentclass[a4paper,12pt,leqno]{article}


% Кодировки; язык
\usepackage[utf8]{inputenc}
\usepackage[T2A]{fontenc}
\usepackage[english, russian]{babel}
% -------------------------------------------------------------------------------------------------



% Поля
\usepackage[left=3cm,right=1.5cm,top=2cm,bottom=2cm]{geometry}
\oddsidemargin=-0.1in
\evensidemargin=-0.1in
\textwidth=6.6in
\topmargin=-0.5in
\textheight=9.1in
% -------------------------------------------------------------------------------------------------



% Математика
\RequirePackage{amsmath, amsfonts, amssymb, amsthm, dsfont}

\usepackage{icomma} % "Умная" запятая: $0,2$ --- число, $0, 2$ --- перечисление


\usepackage{yhmath} % overarc

\RequirePackage{nicefrac, faktor} % красивое деление
% -------------------------------------------------------------------------------------------------


\endinput


% Множества
\newcommand{\N}{\mathbb{N}}
\newcommand{\Z}{\mathbb{Z}}
\newcommand{\Q}{\mathds{Q}}
\newcommand{\R}{\mathds{R}}
\newcommand{\eR}{\overline{\R}} % расширенная вещественная прямая
\newcommand{\Cx}{\mathbb{C}}

% --------------------------------------------------------------------------------------------------------


% новые окружения
\theoremstyle{definition}

\newtheorem{definition}{Определение}[section] % окружение для определений
\newtheorem{theorem}{Теорема}[section] % окружение для теорем
\newtheorem{lemma}{Лемма}[theorem] % окружение для лемм

\newtheorem{corollary}{Следствие}[theorem] % более правильное окружение для следствий

\newtheorem{remark}{Замечание}[section] % окружение для замечаний

\newtheorem{property}{Свойства}[definition] % окружение для свойств

\newtheorem{example}{Пример}[section] % окружение для примеров

\newtheorem{task}{Упражнение}[section] % окружение для упражнений

\renewcommand\qedsymbol{$\blacksquare$} % красивый квадратик в конце доказательства



\endinput


\usepackage{hyperref}
\hypersetup{				% Гиперссылки
	unicode=true,           % русские буквы в раздела PDF
	pdftitle={Доказательство неравенств},   % Заголовок
	pdfauthor={},      % Автор
	pdfsubject={},      % Тема
	pdfcreator={}, % Создатель
	%pdfproducer={Производитель}, % Производитель
	%pdfkeywords={keyword1} {key2} {key3}, % Ключевые слова
	colorlinks=true,       	% false: ссылки в рамках; true: цветные ссылки
	linkcolor=red,          % внутренние ссылки
	citecolor=black,        % на библиографию
	filecolor=magenta,      % на файлы
	urlcolor=cyan           % на URL
}



\title{\textbf{Степенные средние}}
\date{}
\begin{document}

	\parskip=0mm
	\linespread{1}
	\maketitle
	
	\newcounter{zadacha}
	
	\newcommand{\z}{\addtocounter{zadacha}{1}%
		\boxed{\arabic{zadacha}} }
	\section*{На занятии}
	
    \newcommand{\hw}{\addtocounter{zadacha}{1}%
	\text{ДЗ }\boxed{\arabic{zadacha}} }

    \begin{definition}
        Степенным средним степени $r$ положительных чисел $a_1, a_2, \dots , a_n$ называется 
        $$M_r = \left(\frac{a_1^r + a_2^r + \dots + a_n^r}{n}\right)^{1\over r}.$$
        Определим $M_0 = \sqrt[n]{a_1a_2\dots a_n}.$
    \end{definition}


    \begin{itemize}
        \item[\z] Пусть $x_1, \dots , x_n > 0$ и $\alpha > 1$. Докажите, что 
        $$\left(x_1+x_2+\dots +x_n \over n \right)^\alpha \leqslant \frac{x_1^\alpha + x_2^\alpha + \dots + x_n^\alpha}{n}.$$  
        \item[\z] Докажите, что $M_r \leqslant M_s$ при $0 < r < s$ и $r < s < 0$.
        \item[\z] Докажите, что $\lim\limits_{r \to 0} M_r = M_0$. 
        \item[\z] Положительные числа $a, b, c$ удовлетворяют условию $$\frac{1}{a} + \frac{1}{b} + \frac{1}{c} = 3$$
        Докажите неравенство $$\frac{1}{\sqrt{a^3+1}} + \frac{1}{\sqrt{b^3+1}} + \frac{1}{\sqrt{c^3+1}} \leqslant \frac{3}{\sqrt{2}}.$$  
    \end{itemize}

    \begin{definition}
        Пусть $\lambda_1 + \dots + \lambda_n = 1$ и $\lambda_i > 0$. Взвешенным средним степенным называется
        $$M_r^\lambda = (\lambda_1a_1^r + \lambda_2a_2^r + \dots + \lambda_na_n^r)^{\frac{1}{r}}$$
        Определим $M_0^\lambda = a_1^{\lambda_1}a_2^{\lambda_2}\dots a_n^{\lambda_n}.$
    \end{definition}

    \begin{itemize}
        \item[\hw] Пусть $x_1, \dots ,x_n > 0$ и $\alpha > 0$. Докажите, что $M_1^\lambda < M_\alpha^\lambda$. 
        \item[\hw] Докажите, что $M_r^\lambda \leqslant M_s^\lambda$ при $0 < r < s$ и $r < s < 0$. 
        \item[\hw] Докажите, что $\lim\limits_{r \to 0} M_r^\lambda = M_0^\lambda$. 
        \item[\hw] Докажите, что функция $M_r^\lambda$ выпукла на $(-\infty, 0)$ и вогнута на $(0,+\infty)$. 
        \item[\hw] Докажите, что для любых положительных чисел $a, b$ выполнено неравенство
        $$\sqrt{ab} \leqslant \frac{1}{3}\sqrt{\frac{a^2+b^2}{2}} + \frac{2}{3} \frac{2}{\frac{1}{a} + \frac{1}{b}}.$$ 
        \item[\hw] Пусть $a, b, c$ --- положительные действительные числа. Докажите, что 
        $$3(a+b+c) \geqslant 8\sqrt[3]{abc} + \sqrt[3]{\frac{a^3+b^3+c^3}{3}}.$$  
    \end{itemize}

    

	
	
	
	
	
	
	
	
	
	
	
	
	
	
	
	
	
	
	
	
	
	
	
	
	
	
	
\end{document}