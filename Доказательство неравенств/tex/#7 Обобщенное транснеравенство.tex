\documentclass[a4paper,12pt,leqno]{article}


% Кодировки; язык
\usepackage[utf8]{inputenc}
\usepackage[T2A]{fontenc}
\usepackage[english, russian]{babel}
% -------------------------------------------------------------------------------------------------



% Поля
\usepackage[left=3cm,right=1.5cm,top=2cm,bottom=2cm]{geometry}
\oddsidemargin=-0.1in
\evensidemargin=-0.1in
\textwidth=6.6in
\topmargin=-0.5in
\textheight=9.1in
% -------------------------------------------------------------------------------------------------



% Математика
\RequirePackage{amsmath, amsfonts, amssymb, amsthm, dsfont}

\usepackage{icomma} % "Умная" запятая: $0,2$ --- число, $0, 2$ --- перечисление


\usepackage{yhmath} % overarc

\RequirePackage{nicefrac, faktor} % красивое деление
% -------------------------------------------------------------------------------------------------


\endinput


% Множества
\newcommand{\N}{\mathbb{N}}
\newcommand{\Z}{\mathbb{Z}}
\newcommand{\Q}{\mathds{Q}}
\newcommand{\R}{\mathds{R}}
\newcommand{\eR}{\overline{\R}} % расширенная вещественная прямая
\newcommand{\Cx}{\mathbb{C}}

% --------------------------------------------------------------------------------------------------------


% новые окружения
\theoremstyle{definition}

\newtheorem{definition}{Определение}[section] % окружение для определений
\newtheorem{theorem}{Теорема}[section] % окружение для теорем
\newtheorem{lemma}{Лемма}[theorem] % окружение для лемм

\newtheorem{corollary}{Следствие}[theorem] % более правильное окружение для следствий

\newtheorem{remark}{Замечание}[section] % окружение для замечаний

\newtheorem{property}{Свойства}[definition] % окружение для свойств

\newtheorem{example}{Пример}[section] % окружение для примеров

\newtheorem{task}{Упражнение}[section] % окружение для упражнений

\renewcommand\qedsymbol{$\blacksquare$} % красивый квадратик в конце доказательства



\endinput


\usepackage{hyperref}
\hypersetup{				% Гиперссылки
	unicode=true,           % русские буквы в раздела PDF
	pdftitle={Доказательство неравенств},   % Заголовок
	pdfauthor={},      % Автор
	pdfsubject={},      % Тема
	pdfcreator={}, % Создатель
	%pdfproducer={Производитель}, % Производитель
	%pdfkeywords={keyword1} {key2} {key3}, % Ключевые слова
	colorlinks=true,       	% false: ссылки в рамках; true: цветные ссылки
	linkcolor=red,          % внутренние ссылки
	citecolor=black,        % на библиографию
	filecolor=magenta,      % на файлы
	urlcolor=cyan           % на URL
}



\title{\textbf{Обобщенное транснеравенство}}
\date{}
\begin{document}

	\parskip=0mm
	\linespread{1}
	\maketitle
	
	\newcounter{zadacha}
	
	\newcommand{\z}{\addtocounter{zadacha}{1}%
		\boxed{\arabic{zadacha}} }
	\section*{На занятии}
	
    \newcommand{\hw}{\addtocounter{zadacha}{1}%
	\text{ДЗ }\boxed{\arabic{zadacha}} }

    \begin{enumerate}
        \item[\z] [\textit{обобщенное транснеравенство}] Рассмотрим наборы вещественных чисел $x_1 \leqslant x_2 \leaslant \dots \leqslant x_n$ и
        $y_1 \leqslant y_2 \leaslant \dots \leqslant y_n$ и выпуклую функцию $f$. Докажите, что если $u_1, u_2, \dots, u_n$ ---произвольная перестановка из $y_i$, то выполнены неравенства:
        $$f(x_1+y_{n}) + f(x_2+y_{n-1}) + \dots + f(x_n+y_1) \leqslant f(x_1+u_1) + f(x_2+u_2) + \dots + f(x_n+u_n) \leqslant$$ $$\leqslant f(x_1+y_1) + f(x_2+y_2) + \dots + f(x_n+y_n).$$
        
        \item[\z] Выбрав подходящую функцию, докажите, что 
        $$(x_1+y_1)(x_2+y_2)\dots(x_n+y_n) \leqslant (x_1+u_1)(x_2+u_2)\dots (x_n+u_n)\leqslant (x_1+y_n)(x_2+y_{n-1})\dots (x_n+y_1).$$ 
    
        \item[\z] Выбрав подходящую функцию, докажите что
        $$\frac{1}{x_1+y_n} + \frac{1}{x_2+y_{n-1}} + \dots + \frac{1}{x_n+y_1} \leqslant \frac{1}{x_1+u_1} + \frac{1}{x_2+u_{2}} + \dots + \frac{1}{x_n+u_n} \leqslant $$ $$\leqslant \frac{1}{x_1+y_1} + \frac{1}{x_2+y_{2}} + \dots + \frac{1}{x_n+y_n}.$$ 

        \item[\z] Докажите, что для положительных чисел $a_1, a_2, \dots a_n$ выполнено неравенство
        $$\left(1+\frac{a_1^2}{a_2}\right)\left(1+\frac{a_2^2}{a_3}\right)\dots \left(1+\frac{a_n^2}{a_1}\right) \geqslant (1+a_1)(1+a_2)\dots(1+a_n).$$ 

        \item[\z] Докажите, что для положительных чисел $x, y, z$ выполнено неравенство
        $$\sqrt{x+2^x} + \sqrt{y+2^y} + \sqrt{z+2^z} \leqslant \sqrt{y+2^x} + \sqrt{z + 2^y} + \sqrt{x + 2^z}.$$ 

        \item[\z] Докажите, что для неотрицательных положительных чисел $x, y, z$ выполнено неравенство
        $$\sqrt{3x^2 + xy} + \sqrt{3y^2 +yz} + \sqrt{3z^2+zx} \leqslant 2(x+y+z).$$ 

    \end{enumerate}

    

	
	
	
	
	
	
	
	
	
	
	
	
	
	
	
	
	
	
	
	
	
	
	
	
	
	
	
\end{document}