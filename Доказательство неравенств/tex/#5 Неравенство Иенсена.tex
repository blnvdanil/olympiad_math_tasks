\documentclass[a4paper,12pt,leqno]{article}

% Кодировки; язык
\usepackage[utf8]{inputenc}
\usepackage[T2A]{fontenc}
\usepackage[english, russian]{babel}
% -------------------------------------------------------------------------------------------------



% Поля
\usepackage[left=3cm,right=1.5cm,top=2cm,bottom=2cm]{geometry}
\oddsidemargin=-0.1in
\evensidemargin=-0.1in
\textwidth=6.6in
\topmargin=-0.5in
\textheight=9.1in
% -------------------------------------------------------------------------------------------------



% Математика
\RequirePackage{amsmath, amsfonts, amssymb, amsthm, dsfont}

\usepackage{icomma} % "Умная" запятая: $0,2$ --- число, $0, 2$ --- перечисление


\usepackage{yhmath} % overarc

\RequirePackage{nicefrac, faktor} % красивое деление
% -------------------------------------------------------------------------------------------------


\endinput


% Множества
\newcommand{\N}{\mathbb{N}}
\newcommand{\Z}{\mathbb{Z}}
\newcommand{\Q}{\mathds{Q}}
\newcommand{\R}{\mathds{R}}
\newcommand{\eR}{\overline{\R}} % расширенная вещественная прямая
\newcommand{\Cx}{\mathbb{C}}

% --------------------------------------------------------------------------------------------------------


% новые окружения
\theoremstyle{definition}

\newtheorem{definition}{Определение}[section] % окружение для определений
\newtheorem{theorem}{Теорема}[section] % окружение для теорем
\newtheorem{lemma}{Лемма}[theorem] % окружение для лемм

\newtheorem{corollary}{Следствие}[theorem] % более правильное окружение для следствий

\newtheorem{remark}{Замечание}[section] % окружение для замечаний

\newtheorem{property}{Свойства}[definition] % окружение для свойств

\newtheorem{example}{Пример}[section] % окружение для примеров

\newtheorem{task}{Упражнение}[section] % окружение для упражнений

\renewcommand\qedsymbol{$\blacksquare$} % красивый квадратик в конце доказательства



\endinput


\usepackage{hyperref}
\hypersetup{				% Гиперссылки
	unicode=true,           % русские буквы в раздела PDF
	pdftitle={\#n+3 Неравенство Иенсена},   % Заголовок
	pdfauthor={},      % Автор
	pdfsubject={},      % Тема
	pdfcreator={}, % Создатель
	%pdfproducer={Производитель}, % Производитель
	%pdfkeywords={keyword1} {key2} {key3}, % Ключевые слова
	colorlinks=true,       	% false: ссылки в рамках; true: цветные ссылки
	linkcolor=red,          % внутренние ссылки
	citecolor=black,        % на библиографию
	filecolor=magenta,      % на файлы
	urlcolor=cyan           % на URL
}

\title{Неравенство Иенсена}


\date{}

\begin{document}
	
	\maketitle

	\newcounter{num}

	\newcommand{\z}{\addtocounter{num}{1}%
	\boxed{\arabic{num}} }

	\newcommand{\hw}{\addtocounter{num}{1}%
	\text{ДЗ }\boxed{\arabic{num}} }
    
	\begin{definition}
		Множество точек называется \textit{выпуклым}, если все точки отрезка, образуемого двумя любыми точками данного множества, принадлежат множеству.
	\end{definition}

	\begin{definition}
		Суперграфик (надграфик) это множество $G_+$, такое что
		$$G_+ = \lbrace (x, y) \; | \; x \in D_{f},\; y \geqslant f(x) \rbrace $$
	\end{definition}

	\begin{definition}
		Субграфик (подграфик) это множество $G_-$, такое что
		$$G_- = \lbrace (x, y) \; | \; x \in D_{f},\; y \leqslant f(x) \rbrace $$
	\end{definition}

	\begin{definition}

		\begin{enumerate}
			\item[] 
		\end{enumerate}
		Фукнция называется \textit{выпуклой}, если ее суперграфик -- выпуклое множество.

		Функция называется \textit{вогнутой}, если ее субграфик  -- выпуклое множество.
	\end{definition}

	\begin{enumerate}
		\item[\z]  Пусть $a < b$ и $x = \lambda a + (1-\lambda) b$. Покажите, что если $\lambda$ пробегает отрезок $[0, 1]$, то $x$ пробегает отрезок  $[a, b]$.
		\item[\z] Докажите, что функция $y = f(x)$ является выпуклой на промежутке $D$ тогда и только тогда, когда
				$$f(\lambda x_1 + \mu x_2) \leq \lambda f(x_1) + \mu f(x_2).$$ 
				для любых $x_1, x_2 \in D$ и неотрицательных $\lambda$ и $\mu$, удовлетворяющий условию $\lambda + \mu = 1$.
		\item[\z] Докажите, что $y = x^2$ -- выпуклая функция на $\R$, а фукнция $y = \sqrt{x}$ -- вогнутая.
		\item[\z] [\textit{Критерий выпуклости}]. Докажите, что функция $f$ выпуклая на отрезке тогда и только тогда, когда $f^{\prime\prime}(x) \geqslant 0$ для всех $x$ из отрезка. Наоборот, функция $f$ вогнута на ортрезке тогда и только тогда, когда $f^{\prime\prime}(x) \leqslant 0$ для всех $x$ из отрезка.
	 
	\end{enumerate}

	\begin{definition}
		Пусть в точках $(x_1, y_1), (x_2, y_2), \dots , (x_n, y_n)$ расположены массы $m_i > 0$. Центром масс данной системы назвается точка $(x_c, y_c)$, такая что
		$$x_c = {m_1x_1 + m_2x_2 + \dots + m_nx_n \over m_1 + m_2 + \dots + m_n},\;\;\;\; y_c = {m_1y_1 + m_2y_2 + \dots + m_ny_n \over m_1 + m_2 + \dots + m_n}.$$
	\end{definition}

	\begin{enumerate}
		\item[\z]  Одну из точек системы назовем красной, а остальные -- синими. Покажите, что центр масс данной системы совпадает с центром масс двух точек: красной и центра масс всех синих (в котором помещена масса всех синих точек).
		\item[\z] Докажите, что центр масс нескольких точек выпуклого множества также принадлежит этому множеству.
		\item[\z] [Неравенство Иенсена]. Рассмотрим выпуклую функцию $y = f(x)$, определенную на промежутке $D$, и числа $x_1, x_2, \dots , x_n \in D$. Пусть $\lambda_1, \lambda_2, \dots , \lambda_n$ -- положительные числа, сумма которых равна 1. Докажите неравенство Иенсена:
			$$\lambda_1f(x_1) + \lambda_2f(x_2) + \dots + \lambda_nf(x_n) \geqslant f(\lambda_1x_1 + \lambda_2x_2 + \dots + \lambda_nx_n).$$ 
		\item[\z] Используя неравенство Иенсена, докажите неравенство между средними:
		$${a_1+a_2+ \dots + a_n \over n} \leqslant \sqrt{ {a_1^2 + a_2^2 + \dots + a_n^2 \over n}  },$$
		$${a_1+a_2 + \dots + a_n \over n} \geqslant {n \over {1\over a_1} + {1 \over a_2} + \dots + {1\over a_n}}.$$ 	
		\item[\z] Докажите неравенство Коши:
					$$\sqrt[n]{x_1x_2 \dots x_n} \leqslant {x_1+x_2+\dots +x_n \over n}.$$
		\item[\z] С помощью неравенства иенсена докажите неравенство
		$$\sqrt[n]{x_1x_2\dots x_n} \geqslant {n \over \frac{1}{x_1} + \frac{1}{x_2} + \dots + \frac{1}{x_n}}$$  
	\end{enumerate}



\end{document}






