\documentclass[a4paper,12pt,leqno]{article}

%%% Работа с русским языком
\usepackage{cmap}					% поиск в PDF
\usepackage{mathtext} 				% русские буквы в формулах
\usepackage[T2A]{fontenc}			% кодировка
\usepackage[utf8]{inputenc}			% кодировка исходного текста
\usepackage[english,russian]{babel}	% локализация и переносы

\renewcommand{\rmdefault}{cmss}
\renewcommand{\ttdefault}{cmss}

%%% Дополнительная работа с математикой
\usepackage{amsmath,amsfonts,amssymb,amsthm,mathtools} % AMS
\usepackage{icomma} % "Умная" запятая: $0,2$ --- число, $0, 2$ --- перечисление

\newcommand{\R}{\mathbb{R}}
\newcommand{\N}{\mathbb{N}}

%% Номера формул
%\mathtoolsset{showonlyrefs=true} % Показывать номера только у тех формул, на которые есть \eqref{} в тексте.
%\usepackage{leqno} % Нумерация формул слева

%% Свои команды
\DeclareMathOperator{\sgn}{\mathop{sgn}}

%% Перенос знаков в формулах (по Львовскому)
\newcommand*{\hm}[1]{#1\nobreak\discretionary{}
	{\hbox{$\mathsurround=0pt #1$}}{}}

%%% Работа с картинками
\usepackage{graphicx}  % Для вставки рисунков
\graphicspath{{images/}{images2/}}  % папки с картинками
\setlength\fboxsep{3pt} % Отступ рамки \fbox{} от рисунка
\setlength\fboxrule{1pt} % Толщина линий рамки \fbox{}
\usepackage{wrapfig} % Обтекание рисунков текстом

%%% Работа с таблицами
\usepackage{array,tabularx,tabulary,booktabs} % Дополнительная работа с таблицами
\usepackage{longtable}  % Длинные таблицы
\usepackage{multirow} % Слияние строк в таблице

%%% Теоремы
\theoremstyle{plain} % Это стиль по умолчанию, его можно не переопределять.
\newtheorem{theorem}{Теорема}[section]
\newtheorem{proposition}[theorem]{Утверждение}

\theoremstyle{definition} % "Определение"
\newtheorem{corollary}{Следствие}[theorem]
\newtheorem{problem}{Задача}[section]

\theoremstyle{remark} % "Примечание"
\newtheorem*{nonum}{Решение}

%%% Программирование
\usepackage{etoolbox} % логические операторы

%%% Страница
\usepackage{extsizes} % Возможность сделать 14-й шрифт
\usepackage{geometry} % Простой способ задавать поля
\geometry{top=20mm}
\geometry{bottom=20mm}
\geometry{left=20mm}
\geometry{right=20mm}
%
%\usepackage{fancyhdr} % Колонтитулы
% 	\pagestyle{fancy}
%\renewcommand{\headrulewidth}{0pt}  % Толщина линейки, отчеркивающей верхний колонтитул
% 	\lfoot{Нижний левый}
% 	\rfoot{Нижний правый}
% 	\rhead{Верхний правый}
% 	\chead{Верхний в центре}
% 	\lhead{Верхний левый}
%	\cfoot{Нижний в центре} % По умолчанию здесь номер страницы

\usepackage{setspace} % Интерлиньяж
%\onehalfspacing % Интерлиньяж 1.5
%\doublespacing % Интерлиньяж 2
%\singlespacing % Интерлиньяж 1

\usepackage{lastpage} % Узнать, сколько всего страниц в документе.

\usepackage{soul} % Модификаторы начертания

\usepackage{hyperref}
\usepackage[usenames,dvipsnames,svgnames,table,rgb]{xcolor}
\hypersetup{				% Гиперссылки
	unicode=true,           % русские буквы в раздела PDF
	pdftitle={\#1 Алгебраические преобразования},   % Заголовок
	pdfauthor={},      % Автор
	pdfsubject={},      % Тема
	pdfcreator={}, % Создатель
	%pdfproducer={Производитель}, % Производитель
	%pdfkeywords={keyword1} {key2} {key3}, % Ключевые слова
	colorlinks=true,       	% false: ссылки в рамках; true: цветные ссылки
	linkcolor=red,          % внутренние ссылки
	citecolor=black,        % на библиографию
	filecolor=magenta,      % на файлы
	urlcolor=cyan           % на URL
}

\usepackage{csquotes} % Еще инструменты для ссылок

\usepackage[style=authoryear,maxcitenames=2,backend=biber,sorting=nty]{biblatex}

\usepackage{multicol} % Несколько колонок

\usepackage{tikz} % Работа с графикой
\usepackage{pgfplots}
\usepackage{pgfplotstable}

\usepackage{ragged2e}
\usepackage{microtype}


\justifying
\sloppy
\tolerance=500
\hyphenpenalty=10000
\emergencystretch=3em

\title{\textbf{Алгебраические преобразования}}
\date{}
\begin{document}
	\fontsize{14}{16pt}\selectfont
	\parskip=0mm
	\linespread{1}
	\maketitle
	
	\newcounter{zadacha}
	
	\newcommand{\z}{\addtocounter{zadacha}{1}%
		\boxed{\arabic{zadacha}} }
	\section*{На занятии}
	\textbf{\textit{Разложение на множители}}
	\begin{enumerate}
		%kvant
		\item[\z] Докажите, что $x^3 + y^3 \geqslant xy(x+y)$, где $x, y \geq 0$.
		
		\item[\z] Докажите, что $$\frac{1}{1+x^2} + \frac{1}{1+y^2} \leqslant \frac{2}{1+xy}$$ для любых $x, y \in [0; 1]$.
		
		\item[\z] Пусть $0 < a < b$. Докажите, что $$\frac{a}{b} < \frac{a+x}{b+x}$$ для любого $x > 0$.
		
	\end{enumerate}
	
	\textbf{\textit{Выделение полного квадрата}}
	
	\begin{enumerate}
		\item[\z] Докажите неравенство $$x + y \geqslant 2\sqrt{xy}$$ для $x, y > 0$.
		
		\item[\z] Докажите, что $$a^2+b^2 \geqslant ab$$ для любых $a$ и $b$.
		
		\item[\z] Докажите, что $$x^2+y^2+1 \geqslant x + y + xy$$ для любых $x$ и $y$.
	\end{enumerate}
	
	\textbf{\textit{Метод разбиения}}
	
	\begin{enumerate}
		\item[\z] Докажите неравенство $$a+b+c \geqslant \sqrt{ab} + \sqrt{bc} + \sqrt{ca} ,$$
		где $a, b, c > 0$. 
		\newpage
		\item[\z] Докажите неравенство $$(a+b)(b+c)(c+a) \geqslant 8abc ,$$
		где $a, b, c > 0$.
	\end{enumerate}
	
	\textbf{\textit{Метод промежуточной оценки}}
	
	\begin{enumerate}
		\item[\z] Докажите неравенство $$1 < \frac{a}{d + a + b} + \frac{b}{a + b + c} +\frac{c}{b+c+d} + \frac{d}{c+d+a} < 2$$
		для любых положительных $a, b, c, d$.
	\end{enumerate}
	
	\textbf{\textit{Метод подстановки}}
	
	\begin{enumerate}
		\item[\z] Докажите неравенство $$\frac{a}{b} + \frac{b}{a} \geqslant 2$$ для $a, b > 0.$
		
		\item[\z] Докажите неравенство $$\frac{xy}{z} + \frac{yz}{x} + \frac{zx}{y} \geqslant x + y + z$$ для любых $x, y, z > 0$.
	\end{enumerate}
	
	\textbf{\textit{Замена переменной}}
	
	\begin{enumerate}
		\item[\z] (\textit{Неравенство Несбитта}) Докажите неравенство $$\frac{a}{b+c} + \frac{b}{c+a} + \frac{c}{a+b} \geqslant \frac{3}{2}$$
		для $a, b, c > 0$.
		
		\item[\z] Докажите неравенство $$x^3 -4x^2-3x+19 > 0 ,$$
		если $x \geqslant 3$.
		
		\item[\z] Докажите неравенство $$-\frac{1}{2} \leqslant \frac{(x+y)(1-xy)}{(1+x^2)(1+y^2)}\leqslant \frac{1}{2}$$
		для любых $x, y$.
	\end{enumerate}
	\newpage
	\textbf{\textit{Метод упорядочивания переменных}}
	
	\begin{enumerate}
		\item[\z] Докажите, что $$\frac{x}{1+y} + \frac{y}{1+x} \leqslant 1 ,$$ где $x, y \in [0; 1].$ 
	\end{enumerate}
	
	\section*{Домашка}
	\textbf{\textit{Докажите неравенства}}
	
	\newcommand{\ner}[1]{\item[\z] #1}
	
	\begin{enumerate}
		\ner{ $2(x^3 + y^3) \geqslant (x+y)(x^2+y^2)$, где $x, y > 0$.}
		
		\ner{$(a^2-b^2)(a^4-b^4) \leqslant (a^3-b^3)$, где $a, b \in \R$}
		
		\ner{$\displaystyle \sqrt{a} + \sqrt{b} \leqslant \sqrt{\frac{a^2}{b}} + \sqrt{\frac{b^2}{a}}$ для любых $a, b > 0$.}
		
		\ner{$\displaystyle\frac{a+b}{2} \cdot \frac{a^3+b^3}{2} \leqslant \frac{a^4+b^4}{2}$ для любых $a, b$.}
		
		\ner{$\displaystyle \frac{1}{a} + \frac{1}{b} \geqslant \frac{4}{a+b}$ для любых $a, b > 0$.}
		
		\ner{$\displaystyle\frac{a^2+b^2}{a+b} \geqslant \frac{a+b}{2} $ для любых $a, b > 0$.}
		
		\ner{$x^4+y^4+z^2+1\geqslant 2x(xy^2-x+z+1)$ для любых $z, y, z$.}
		
		\ner{$a^2 + b^2 +c^2 \geqslant ab+bc+ca$ для любых $x, y, z$.}
		
		\ner{$(x+y+z)^2\geqslant 3(xy+yz+zx)$ для любых $x, y, z$.}
		
		\ner{$a^2 + ab + b^2 \geqslant 3(a+b-1)$ для любых $a, b$.}
		
		\ner{$(x^2 + y^2+z^2)(xy+yz+zx)\geqslant 2(x^2y^2 + y^2z^2 +z^2x^2)$, где $x, y, z > 0$.}
		
		\ner{$(x+y+z)^4 \geqslant 8(xy+yz+zx)(x^2+y^2+z^2)$, где $x, y, z > 0$.}
		
		\ner{$\displaystyle\frac{1}{a^2+b^2} + \frac{1}{b^2+c^2} +\frac{1}{c^2+a^2} \leqslant \frac{a+b+c}{2abc}$, где $a, b, c > 0$.}
		
		\ner{$\displaystyle\frac{1}{1+x} +\frac{1}{1+y} \leqslant \frac{3}{1+x+y}$, где $x, y \in [0;1]$.}
		
		\ner{$\displaystyle1 + \frac{1}{\sqrt{2}} + \dots + \frac{1}{\sqrt{n}} \geqslant \sqrt{n}$ для каждого $n \in \N$.}
		\newpage
		\ner{ $\displaystyle x_0, x_1, \dots, x_n >0, \;\; n \geqslant 2$ $$\displaystyle \frac{x_1}{x_0 + x_1} + \frac{x_2}{x_0 + x_2} + \dots + \frac{x_n}{x_0 + x_n} > \frac{x_1+x_2+\dots+x_n}{x_0+x_1+\dots+x_n},$$}
		
		\ner{$\displaystyle\frac{1}{1^2} + \frac{1}{2^2} +\frac{1}{3^2} + \dots + \frac{1}{n^2} < 2$ для каждого $n \in \N$.}
		
		\ner{$\displaystyle\frac{1}{2} \cdot \frac{3}{4} \cdot \dots \cdot \frac{2n-1}{2n} < \frac{1}{\sqrt{2n+1}} $ для каждого натурального $n$.}
		
		\ner{$(ab+bc+ca)^2 \geqslant 3abc(a+b+c)$ для любых $a, b, c$.}
		
		\ner{$\displaystyle\frac{1}{a} + \frac{1}{b} + \frac{1}{c} \geqslant \frac{3(a+b+c)}{ab+bc+ca}$ для любых $a, b, c > 0$.}
		
		\ner{$a^4+b^4+c^4 \geqslant abc(a+b+c)$ для любых $a, b, c$.}
		
		\ner{$\sqrt{(x+y)(y+z)} + \sqrt{(x+z)(z+x)} \sqrt{(z+x)(x+y)} \leqslant 2(x+y+z)$, где $x, y, z \geq 0$.}
		
		\ner{$\displaystyle\frac{x^3}{yz} + \frac{y^3}{zx} + \frac{z^3}{xy} \geqslant \frac{xy}{z} + \frac{yz}{x} + \frac{zx}{y}$ для любых $x, y, z > 0$.}
		
		\ner{$\displaystyle\frac{x^2}{y^2} + \frac{y^2}{z^2} + \frac{z^2}{x^2} \geqslant \frac{x}{z} + \frac{y}{x} + \frac{z}{y}$ для любых $x, y, z > 0$.}
		
		\ner{$2x^3+3y^3 \geqslant 4xy^2$ для любых $x, y > 0$.}
		
		\ner{$(x^a +y^a+z^a)(x^d+y^d+z^d) \geqslant (x^b+y^b +z^b)(x^c+y^c+z^c)$ для любых $x, y, z \geqslant 0$, где $0 < a < b < c < d$ и $a+d = b+c$.}
		
	\end{enumerate}
	
	
	
	
	
	
	
	
	
	
	
	
	
	
	
	
	
	
	
	
	
	
	
	
	
	
	
	
\end{document}