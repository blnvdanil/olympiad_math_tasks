\documentclass[a4paper,12pt,leqno]{article}

%%% Работа с русским языком
\usepackage{cmap}					% поиск в PDF
\usepackage{mathtext} 				% русские буквы в формулах
\usepackage[T2A]{fontenc}			% кодировка
\usepackage[utf8]{inputenc}			% кодировка исходного текста
\usepackage[english,russian]{babel}	% локализация и переносы

\renewcommand{\rmdefault}{cmss}
\renewcommand{\ttdefault}{cmss}

%%% Дополнительная работа с математикой
\usepackage{amsmath,amsfonts,amssymb,amsthm,mathtools} % AMS
\usepackage{icomma} % "Умная" запятая: $0,2$ --- число, $0, 2$ --- перечисление

\newcommand{\R}{\mathbb{R}}
\newcommand{\N}{\mathbb{N}}

%% Номера формул
%\mathtoolsset{showonlyrefs=true} % Показывать номера только у тех формул, на которые есть \eqref{} в тексте.
%\usepackage{leqno} % Нумерация формул слева

%% Свои команды
\DeclareMathOperator{\sgn}{\mathop{sgn}}

%% Перенос знаков в формулах (по Львовскому)
\newcommand*{\hm}[1]{#1\nobreak\discretionary{}
	{\hbox{$\mathsurround=0pt #1$}}{}}

%%% Работа с картинками
\usepackage{graphicx}  % Для вставки рисунков
\graphicspath{{images/}{images2/}}  % папки с картинками
\setlength\fboxsep{3pt} % Отступ рамки \fbox{} от рисунка
\setlength\fboxrule{1pt} % Толщина линий рамки \fbox{}
\usepackage{wrapfig} % Обтекание рисунков текстом

%%% Работа с таблицами
\usepackage{array,tabularx,tabulary,booktabs} % Дополнительная работа с таблицами
\usepackage{longtable}  % Длинные таблицы
\usepackage{multirow} % Слияние строк в таблице

%%% Теоремы
\theoremstyle{plain} % Это стиль по умолчанию, его можно не переопределять.
\newtheorem{theorem}{Теорема}[section]
\newtheorem{proposition}[theorem]{Утверждение}

\theoremstyle{definition} % "Определение"
\newtheorem{corollary}{Следствие}[theorem]
\newtheorem{problem}{Задача}[section]

\theoremstyle{remark} % "Примечание"
\newtheorem*{nonum}{Решение}

%%% Программирование
\usepackage{etoolbox} % логические операторы

%%% Страница
\usepackage{extsizes} % Возможность сделать 14-й шрифт
\usepackage{geometry} % Простой способ задавать поля
\geometry{top=20mm}
\geometry{bottom=20mm}
\geometry{left=20mm}
\geometry{right=20mm}
%
%\usepackage{fancyhdr} % Колонтитулы
% 	\pagestyle{fancy}
%\renewcommand{\headrulewidth}{0pt}  % Толщина линейки, отчеркивающей верхний колонтитул
% 	\lfoot{Нижний левый}
% 	\rfoot{Нижний правый}
% 	\rhead{Верхний правый}
% 	\chead{Верхний в центре}
% 	\lhead{Верхний левый}
%	\cfoot{Нижний в центре} % По умолчанию здесь номер страницы

\usepackage{setspace} % Интерлиньяж
%\onehalfspacing % Интерлиньяж 1.5
%\doublespacing % Интерлиньяж 2
%\singlespacing % Интерлиньяж 1

\usepackage{lastpage} % Узнать, сколько всего страниц в документе.

\usepackage{soul} % Модификаторы начертания

\usepackage{hyperref}
\usepackage[usenames,dvipsnames,svgnames,table,rgb]{xcolor}
\hypersetup{				% Гиперссылки
	unicode=true,           % русские буквы в раздела PDF
	pdftitle={\#n Классические неравенства},   % Заголовок
	pdfauthor={},      % Автор
	pdfsubject={},      % Тема
	pdfcreator={}, % Создатель
	%pdfproducer={Производитель}, % Производитель
	%pdfkeywords={keyword1} {key2} {key3}, % Ключевые слова
	colorlinks=true,       	% false: ссылки в рамках; true: цветные ссылки
	linkcolor=red,          % внутренние ссылки
	citecolor=black,        % на библиографию
	filecolor=magenta,      % на файлы
	urlcolor=cyan           % на URL
}

\usepackage{csquotes} % Еще инструменты для ссылок

\usepackage[style=authoryear,maxcitenames=2,backend=biber,sorting=nty]{biblatex}

\usepackage{multicol} % Несколько колонок

\usepackage{tikz} % Работа с графикой
\usepackage{pgfplots}
\usepackage{pgfplotstable}

\usepackage{ragged2e}
\usepackage{microtype}


\justifying
\sloppy
\tolerance=500
\hyphenpenalty=10000
\emergencystretch=3em

\title{\textbf{Классические неравенства}}
\date{}
\begin{document}
	\fontsize{14}{16pt}\selectfont
	\parskip=0mm
	\linespread{1}
	\maketitle
	
	\newcounter{zadacha}
	
	\newcommand{\z}{\addtocounter{zadacha}{1}%
		\boxed{\arabic{zadacha}} }
	\section*{На занятии}
	\textbf{\textit{Неравенство КБШ и лемма Титу}}
	\begin{enumerate}
		%бибиков
		\item[\z] Докажите неравенство  КБШ
		$$(x_1^2 + x_2^2 + \dots + x_n^2)\cdot (y_1^2+y_2^2+ \dots + y_n^2) \geqslant (x_1y_1 + x_2y_2 + \dots + x_ny_n)^2 .$$
		
		\item[\z] Докажите неравенство [\textit{лемма Титу}] ($y_i > 0$)
		$$\frac{x_1^2}{y_1} + \frac{x_2^2}{y_2} + \dots + \frac{x_n^2}{y_n} \geqslant \frac{(x_1+x_2+ \dots + x_n)^2}{y_1+y_2+ \dots + y_n}.$$
		%5.2 Бибиков 
		\item[\z] Докажите, что для любых $a, b, c > 0$ верно неравенство
		$$\frac{a}{b} + \frac{b}{c} + \frac{c}{a} \geqslant \frac{(a+b+c)^2}{ab + bc+ca}.$$
		%5.3
		\item[\z] Докажите, что для любых $a, b, c > 0$ верно неравенство
		$$\frac{a}{a+2b} + \frac{b}{b+2c}+ \frac{c}{c+2a} \geqslant 1.$$
		%5.4
		\item[\z] Докажите, что для любых $a, b, c > 0$ выполняется неравенство
		$$\left( \frac{a}{a+2b} \right) ^ 2 + \left( \frac{b}{b+2c}\right) ^2 + \left( \frac{c}{c+2a}\right)^2 \geqslant \frac{1}{3}  .$$
		%5.5
		\item[\z] Пусть $a, b, c > 0, \;\; abc = 1$. Докажите, что 
		$$\frac{1}{a^3(b+c)} + \frac{1}{b^3(c+a)} + \frac{1}{c^3(a+b)} \geqslant \frac{3}{2}.$$  
		%5.6
		\item[\z] Пусть $a, b, c > 0, \;\; abc=1$. Докажите, что 
		$$\frac{a^3}{b+c} + \frac{b^3}{c+a} + \frac{c^3}{a+b} \geqslant \frac{3}{2}.$$
		%5.7
		\item[\z] $a, b, c > 0$. Докажите, что 
		$$\frac{a^2}{(a+b)(a+c)} + \frac{b^2}{(b+c)(b+a)} + \frac{c^2}{(c+a)(c+b)} \geqslant \frac{3}{4}.$$
	\end{enumerate}
	
	\textbf{\textit{Неравенство Гёльдера и обобщенная лемма Титу}}
	
	\begin{enumerate}
		
		\item[\z] [\textit{Неравенство Гёльдера}] Докажите, что для трех наборов положительных чисел $(a_1, a_2, \dots , a_n),  (b_1, b_2, \dots , d_n)$ и $(c_1, c_2, \dots c_n)$ справедливо неравенство
		$$(a_1^3 + a_2^3 + \dots + a_n^3) \cdot (b_1^3 + b_2^3 + \dots + b_n^3) \cdot (c_1^3 + c_2^3 + \dots + c_n^3) \geqslant (a_1b_1c_1 + a_2b_2c_2 + \dots + a_nb_nc_n)^3.$$
		
		\item[\z] [\textit{Неравенство Гёльдера}] Докажите, что для $k$ наборов положительных чисел $(a_{11}, a_{12}, \dots , a_{1n}), (a_{21}, a_{22}, \dots , a_{2n}), \dots , (a_{k1}, a_{k2}, \dots , a_{kn})$ справедливо неравенство
		$$\prod\limits_{i = 1}^{k} \left(\sum\limits_{j = 1}^n a_{ij}^k \right) \geqslant \left( \sum\limits_{j = 1} \prod\limits_{i = 1}^n a_{ij} \right)^k .$$
		
		\item[\z] [\textit{Обобщенная лемма Титу}] Докажите, что для любых положительных чисел $(x_1, x_2, \dots , x_n)$ и $(y_1, y_2, \dots, y_n)$ выполнено неравенство $$\frac{x_1^3}{y_1} +\frac{x_2^3}{y_2} + \dots + \frac{x_n^3}{y_n} \geqslant \frac{(x_1 + x_2 + \dots + x_n)^3}{n(y_1 + y_2 + \dots + y_n)}.$$
		
		\item[\z] Докажите неравенство для любых положительных $a, b, c, \; a+b+c = 1$ $$\frac{1}{a+b} + \frac{16}{c} + \frac{81}{a+b+c} \geqslant 98.$$
		% bibikov 5.11
		\item[\z] Докажите, что для любых $a, b, c > 0$ выполнено неравенство $$\frac{a^6}{b^2+c^2} + \frac{b^6}{c^2+a^2} + \frac{c^6}{a^2+b^2} \geqslant \frac{1}{2}abc(a+b+c).$$
		% bibikov 5.12
		\item[\z] Пусть $x, y, z$ такие положительные числа, что $xy + yz +zx = 1$. Докажите, что $$\frac{x^3}{1+9y^2xz} + \frac{y^3}{1+9z^2xy} + \frac{z^3}{1+9x^2yz} \geqslant \frac{(x+y+z)^3}{18}.$$
		
		%bibikov 5.13
		
		\item[\z] Докажите, что для всех положительных действительных чисел $a, b, c$ выполнено нераевнство $$\frac{1}{a(b+c)} + \frac{1}{b(c+a)} + \frac{1}{c(a+b)} \geqslant \frac{27}{2(a+b+c)^2}.$$
		
		
	\end{enumerate}
	
	
	
	
	
	
	
	
	
	
	
	
	
	
	
	
	
	
	
	
	
	
	
	
	
	
	
	
\end{document}