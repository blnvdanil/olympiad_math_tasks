\documentclass[a4paper,12pt,leqno]{article}

%%% Работа с русским языком
\usepackage{cmap}					% поиск в PDF
\usepackage{mathtext} 				% русские буквы в формулах
\usepackage[T2A]{fontenc}			% кодировка
\usepackage[utf8]{inputenc}			% кодировка исходного текста
\usepackage[english,russian]{babel}	% локализация и переносы

\renewcommand{\rmdefault}{cmss}
\renewcommand{\ttdefault}{cmss}

%%% Дополнительная работа с математикой
\usepackage{amsmath,amsfonts,amssymb,amsthm,mathtools} % AMS
\usepackage{icomma} % "Умная" запятая: $0,2$ --- число, $0, 2$ --- перечисление

\newcommand{\R}{\mathbb{R}}
\newcommand{\N}{\mathbb{N}}

%% Номера формул
%\mathtoolsset{showonlyrefs=true} % Показывать номера только у тех формул, на которые есть \eqref{} в тексте.
%\usepackage{leqno} % Нумерация формул слева

%% Свои команды
\DeclareMathOperator{\sgn}{\mathop{sgn}}

%% Перенос знаков в формулах (по Львовскому)
\newcommand*{\hm}[1]{#1\nobreak\discretionary{}
	{\hbox{$\mathsurround=0pt #1$}}{}}

%%% Работа с картинками
\usepackage{graphicx}  % Для вставки рисунков
\graphicspath{{images/}{images2/}}  % папки с картинками
\setlength\fboxsep{3pt} % Отступ рамки \fbox{} от рисунка
\setlength\fboxrule{1pt} % Толщина линий рамки \fbox{}
\usepackage{wrapfig} % Обтекание рисунков текстом

%%% Работа с таблицами
\usepackage{array,tabularx,tabulary,booktabs} % Дополнительная работа с таблицами
\usepackage{longtable}  % Длинные таблицы
\usepackage{multirow} % Слияние строк в таблице

%%% Теоремы
\theoremstyle{plain} % Это стиль по умолчанию, его можно не переопределять.
\newtheorem{theorem}{Теорема}[section]
\newtheorem{proposition}[theorem]{Утверждение}

\theoremstyle{definition} % "Определение"
\newtheorem{corollary}{Следствие}[theorem]
\newtheorem{problem}{Задача}[section]

\theoremstyle{remark} % "Примечание"
\newtheorem*{nonum}{Решение}

%%% Программирование
\usepackage{etoolbox} % логические операторы

%%% Страница
\usepackage{extsizes} % Возможность сделать 14-й шрифт
\usepackage{geometry} % Простой способ задавать поля
\geometry{top=20mm}
\geometry{bottom=20mm}
\geometry{left=20mm}
\geometry{right=20mm}
%
%\usepackage{fancyhdr} % Колонтитулы
% 	\pagestyle{fancy}
%\renewcommand{\headrulewidth}{0pt}  % Толщина линейки, отчеркивающей верхний колонтитул
% 	\lfoot{Нижний левый}
% 	\rfoot{Нижний правый}
% 	\rhead{Верхний правый}
% 	\chead{Верхний в центре}
% 	\lhead{Верхний левый}
%	\cfoot{Нижний в центре} % По умолчанию здесь номер страницы

\usepackage{setspace} % Интерлиньяж
%\onehalfspacing % Интерлиньяж 1.5
%\doublespacing % Интерлиньяж 2
%\singlespacing % Интерлиньяж 1

\usepackage{lastpage} % Узнать, сколько всего страниц в документе.

\usepackage{soul} % Модификаторы начертания

\usepackage{hyperref}
\usepackage[usenames,dvipsnames,svgnames,table,rgb]{xcolor}
\hypersetup{				% Гиперссылки
	unicode=true,           % русские буквы в раздела PDF
	pdftitle={\#n Классические неравенства},   % Заголовок
	pdfauthor={},      % Автор
	pdfsubject={},      % Тема
	pdfcreator={}, % Создатель
	%pdfproducer={Производитель}, % Производитель
	%pdfkeywords={keyword1} {key2} {key3}, % Ключевые слова
	colorlinks=true,       	% false: ссылки в рамках; true: цветные ссылки
	linkcolor=red,          % внутренние ссылки
	citecolor=black,        % на библиографию
	filecolor=magenta,      % на файлы
	urlcolor=cyan           % на URL
}

\usepackage{csquotes} % Еще инструменты для ссылок

\usepackage[style=authoryear,maxcitenames=2,backend=biber,sorting=nty]{biblatex}

\usepackage{multicol} % Несколько колонок

\usepackage{tikz} % Работа с графикой
\usepackage{pgfplots}
\usepackage{pgfplotstable}

\usepackage{ragged2e}
\usepackage{microtype}


\justifying
\sloppy
\tolerance=500
\hyphenpenalty=10000
\emergencystretch=3em

\title{\textbf{Классические неравенства}}
\date{}
\begin{document}
	\fontsize{14}{16pt}\selectfont
	\parskip=0mm
	\linespread{1}
	\maketitle
	
	\newcounter{zadacha}
	
	\newcommand{\z}{\addtocounter{zadacha}{1}%
		\boxed{\arabic{zadacha}} }
	\section*{Домашка}
	\textbf{\textit{Докажите неравенства}}
	
	\newcommand{\ner}[1]{\item[\z] #1}
	
	\begin{enumerate}
		%kvant
		\ner{$(x^3+y^3+z^3)(x+y+z) \geqslant (x^2+y^2+z^2)^2$ для любых $x, y, z > 0.$}
		
		\ner{[\textit{Неравенство Несбитта}] $\displaystyle\frac{a}{b+c} + \frac{c}{c+a} + \frac{c}{a+b} \geqslant \frac{3}{2}$ для $a, b, c > 0.$}
		
		\ner{$1+ab \leqslant \sqrt{1+a^2} \cdot \sqrt{1+b^2}$ для любых $a, b$.}
		
		\ner{$\sqrt{(a+c)(b+d)} \geqslant \sqrt{ab} + \sqrt{cd}$ для любых $a, b, c, d > 0$.}
		
		\ner{$(a^2+b^2)(a^4+b^4) \geqslant (a^3+b^3)^2$ для любых $a, b, c$.}
		
		\ner{$\sin x \sin y \sin z + \cos x \cos y \cos z \leqslant 1$ для любых $x, y, z$.}
		
		\ner{$\displaystyle(a_1b_1 + \dots +a_nb_n) \left( \frac{a_1}{b_1} + \dots + \frac{a_n}{b_n} \right) \geqslant (a_1+\dots+a_n)^2 $ где $a_i, b_i > 0$.}
		
		\ner{$\displaystyle(\sqrt{a_1b_1} + \dots + \sqrt{a_nb_n})^2 \leqslant (a_1c_1 + \dots + a_nc_n)\left( \frac{b_1}{c_1} + \dots + \frac{b_n}{c_n}\right) $ для любых $a_i, b_i< c_i > 0.$}
		
		\ner{$x_1x_2 + x_2x_3 + \dots + x_nx_1 \leqslant x_1^2 + x_2^2 + \dots + x_n^2$ для любых действительных $x_i$.}
		
		\ner{[\textit{Неравенство треугольника}] $\sqrt{(a_1 - b_1)^2 + (a_2 - b_2)^2 + \dots + (a_n-b_n)^2} \leqslant \sqrt{a_1^2 + a_2^2 + \dots + a_n^2} + \sqrt{b_1^2 +b_2^2 + \dots + b_n^2}$ для любых $a_i, b_i$.}
		
		\ner{$\displaystyle\frac{(x+y+z)^2}{x^2+y^2+z^2} \leqslant \frac{x}{y} + \frac{y}{z} + \frac{z}{x}$ для любых $x, y, z > 0$.}
		
		\ner{$\displaystyle\frac{a}{b+2c+d} + \frac{b}{c+2d+a} + \frac{c}{d+2a+b} + \frac{d}{a+2b+c} \geqslant 1$, где $a, b, c, d > 0$.}
		
		\ner{$\displaystyle\frac{a^2}{b(a+c)} + \frac{b^2}{c(b+d)} + \frac{c^2}{d(c+a)} + \frac{d^2}{a(d+b)} \geqslant 2$, если $a, b, c, d > 0$.}
		
		\ner{$\displaystyle\frac{a^3}{a^2+ab+b^2} + \frac{b^3}{b^2+bc+c^2} + \frac{c^3}{c^2+ca+a^2} \geqslant \frac{a+b+c}{3}$}
		
		 для любых положительных $a, b, c$.
		
		\ner{$\displaystyle\frac{ab}{a+b} +\frac{bc}{b+c} + \frac{ca}{c+a} \geqslant \frac{(ab + bc+ ca)(a+b+c)}{2(a^2+b^2+c^2)}$ для любых $a, b, c > 0$.}
		
		\ner{$\displaystyle\frac{x}{y} + \frac{y}{z} + \frac{z}{x} \geqslant \sqrt{\frac{x^2+y^2+z^2}{3}}$ если $x, y, z > 0$ и $x+y+z = 3$.}
		
		\ner{$\displaystyle\sqrt{a-1} + \sqrt{b-1} + \sqrt{c-1} \leqslant \frac{2}{\sqrt{3}} \sqrt{abc}$ для $a, b, c > 1$}
		
		\ner{$\displaystyle\sqrt{a+b+c} \geqslant \frac{a}{\sqrt{2a+b}} +\frac{b}{\sqrt{2b + c}} + \frac{c}{\sqrt{2c+a}}$ для любых $a, b, c > 0$.}
		\addtolength{\parskip}{0.2cm}
		\ner{$\displaystyle\frac{ac}{ax^2 +2bx + c} + \frac{ba}{bx^2+2cx+a}+\frac{cb}{cx^2+2ax+b} \leqslant \frac{a+b+c}{(x+1)^2}$, где}
		
		все переменные неотрицательны и никакие две из них не равны нулю.
		\addtolength{\parskip}{-0.2cm}
		
		\ner{$\displaystyle\frac{a}{a+b} + \frac{b}{b+c} + \frac{c}{c+a} \geqslant \frac{a+b+c}{a+b+c - \sqrt[3]{abc}}$ для любых $a, b, c > 0$.}
		
	\end{enumerate}
	
	
	
	
	
	
	
	
	
	
	
	
	
	
	
	
	
	
	
	
	
	
	
	
	
	
	
	
	
\end{document}